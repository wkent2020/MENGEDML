\documentclass[12pt]{amsart}
\usepackage[margin=1in, letterpaper]{geometry}
\usepackage[utf8]{inputenc}
\def\labelitemi{--}
\usepackage[english]{babel}
\usepackage{siunitx}
\usepackage{graphicx}
\graphicspath{ {./images/} }
\DeclareGraphicsExtensions{.pdf,.png,.jpg}
\usepackage{fancyhdr}
\usepackage{csquotes}
\usepackage{fixltx2e}
\setlength{\parskip}{11pt}
\setlength{\parindent}{0cm}
\usepackage{braket}
\usepackage{cancel}

\usepackage[utf8]{inputenc}
\usepackage{tikz}
\usetikzlibrary{shapes,arrows}

\usepackage{amsmath}
\DeclareMathOperator{\sech}{sech}
\DeclareMathOperator{\csch}{csch}
\DeclareMathOperator{\arcsec}{arcsec}
\DeclareMathOperator{\arccot}{arcCot}
\DeclareMathOperator{\arccsc}{arcCsc}
\DeclareMathOperator{\arccosh}{arcCosh}
\DeclareMathOperator{\arcsinh}{arcsinh}
\DeclareMathOperator{\arctanh}{arctanh}
\DeclareMathOperator{\arcsech}{arcsech}
\DeclareMathOperator{\arccsch}{arcCsch}
\DeclareMathOperator{\arccoth}{arcCoth}

\usepackage{caption}
\usepackage{subcaption}
% \captionof{table}{CAPTION}

\setlength{\tabcolsep}{6pt}
\renewcommand{\arraystretch}{1.2}
% \renewcommand{\arraystretch}{0.7} WHEN USING DOUBLE SPACING

\usepackage[utf8]{inputenc}
 
\usepackage{listings}
\usepackage{color}

\usepackage{courier}
\definecolor{codegreen}{rgb}{0,0.6,0}
\definecolor{codegray}{rgb}{0.5,0.5,0.5}
\definecolor{codepurple}{rgb}{0.58,0,0.82}
\definecolor{backcolour}{rgb}{0.95,0.95,0.92}
 
\lstdefinestyle{mystyle}{
    backgroundcolor=\color{backcolour},   
    commentstyle=\color{codegreen},
    keywordstyle=\color{magenta},
    numberstyle=\tiny\color{codegray},
    stringstyle=\color{codepurple},
    basicstyle=\footnotesize\ttfamily,
    breakatwhitespace=false,         
    breaklines=true,                 
    captionpos=b,                    
    keepspaces=true,                 
    numbers=left,                    
    numbersep=5pt,                  
    showspaces=false,                
    showstringspaces=false,
    showtabs=false,                  
    tabsize=2,
}
 
\lstset{style=mystyle}

\def\labelitemi{--}

\pagestyle{fancy}
\fancyhf{}
\fancyhead[L]{ \textsc{Week 3 Plan}}
\fancyhead[R]{\textsc{Machine Learning Team}}
\fancyfoot[LE,RO]{\thepage}

\fancypagestyle{firststyle}
{
\fancyhf{}
\fancyhead[C]{\textsc{Chemistry 121 Problem Set \#1}}
\fancyhead[L]{\textsc{Will Kent}}
\fancyhead[R]{\textsc{\today}}
\renewcommand{\headrulewidth}{0pt}
}

\usepackage[version=3]{mhchem}
\usepackage{siunitx}
\sisetup{detect-weight=true, detect-family=true} % Allows siunitx to be bold in text environment

\usepackage{array}
\usepackage{tabu}

\usepackage{multicol}

\usepackage{setspace}

\title{Machine Learning Team: Week 3 Plan}
\author{Will Kent, Jackson Dougherty, Keir Adams}
\date{\today}

\begin{document}
\maketitle

Last week the three of us researched different ways to process, cluster, threshold, and contour images. We have a general flow chart on the next page outlining the whole process. We have come up with several options for thresholding and image processing that we believe offer great improvement over the methods used by the previous group. We should be able to test all of these techniques by the end of next week. 

The first general problem that we are faced with in this project is thresholding images. We need to isolate the droplet shapes in the image in order to perform machine learning analysis on the images. The validity of our machine learning methods is dependent upon how much information our thresholding and contouring methods obtain. 

The methods we will try for thresholding are listed below:

Thresholding:
\vspace*{-11pt}
\begin{itemize}
	\item $k$-means
	\item Segmented $k$-means (segmenting the images and performing k-means)
	\item 3D $k$-means ($k$-means applied to color and position information)
	\item High $k$ $k$-means (Applying very large $k$ values for better thresholding)
	\item Additional methods
\end{itemize}

We are also going to see if performing thresholding analysis on images within the 16-bit encoding, as opposed to the 8-bit encoding used in the normalized images. We will also be researching and applying other methods besides k-means to cluster and threshold the data. Furthermore, we have found some new image preprocessing methods that increase contrast within the images. These methods are easy to implement and may improve the performance of our thresholding. 




\begin{center}
% Define block styles
\tikzstyle{decision} = [diamond, draw, fill=green!20, 
    text width=4.5em, text badly centered, node distance=3cm, inner sep=0pt]
\tikzstyle{block} = [rectangle, draw, fill=blue!20, 
    text width=8em, text centered, rounded corners, minimum height=4em, node distance=2cm]
 \tikzstyle{gpio} = [rectangle, draw, fill=red!20, 
    text width=8em, text centered, rounded corners, minimum height=4em, node distance=2cm]
\tikzstyle{line} = [draw, -latex']
\tikzstyle{cloud} = [draw, ellipse,fill=red!20, node distance=3cm,
    minimum height=2em]
    
\begin{tikzpicture}[node distance = 2cm, auto]
    % Place nodes
    \node [gpio] (init) {Raw Image};
    \node [decision, below of=init, node distance=2.5cm, text width=2cm] (norm) {Normalize?};
    \node [block, right of=norm, node distance=4cm] (normalize) {Normalize};
    \node [block, below of=norm, node distance = 2.5cm] (removeback) {Remove Background};
    \node [decision, below of=removeback] (decidefilters) {Apply Additional Filters?};
    \node [block, right of=decidefilters, node distance=5cm] (applyfilter) {Apply Filter};
    \node [gpio, below of=decidefilters, node distance=3.5cm] (finishedimage) {Final Processed Image};
    \node [decision, below of=finishedimage, node distance=3.5cm] (whichthresh) {Which Thresholding Technique?};
    \node [block, below of=whichthresh, node distance=3.5cm] (segkmeans) {Segmented K-means Thresholding};
    \node  [block, left of=segkmeans, node distance=4cm] (kmeans) {Regular K-means Thresholding};
     \node  [block, right of=segkmeans, node distance=4cm] (3dkmeans) {3D K-means Thresholding};
     \node  [block, right of=3dkmeans, node distance=4cm] (notkmeans) {Non K-means Thresholding};
     \node  [gpio, below of=segkmeans, node distance=2.5cm] (threshim) {Thresholded Image};
    % Draw edges
    \path [line] (init) -- (norm);
    \path [line] (norm) --  node {Yes} (normalize);
    \path [line] (normalize) |- (removeback);
    \path [line] (norm) -- node {No} (removeback);
    \path [line] (removeback) -- (decidefilters);
    \path [line] (decidefilters) -r node {Yes} (applyfilter);
    \path [line] (applyfilter) |- (finishedimage);
    \path [line] (decidefilters) -- node {No}(finishedimage);
    \path [line] (finishedimage) -- (whichthresh);
    \path [line] (whichthresh) -| (kmeans);
    \path [line] (whichthresh) -- (segkmeans);
    \path [line] (whichthresh) -| (notkmeans);
    \path [line] (whichthresh) -| (3dkmeans);
    \path [line] (segkmeans) -- (threshim);
    \path [line] (kmeans) |- (threshim);
    \path [line] (3dkmeans) |- (threshim);
    \path [line] (notkmeans) |- (threshim);
\end{tikzpicture}
\end{center}



\end{document}

